% \iffalse
% $Id$
%
%% Copyright 2015, Boris Veytsman <borisv@lk.net
%% This work may be distributed and/or modified under the
%% conditions of the LaTeX Project Public License, either
%% version 1.3 of this license or (at your option) any 
%% later version.
%% The latest version of the license is in
%%    http://www.latex-project.org/lppl.txt
%% and version 1.3 or later is part of all distributions of
%% LaTeX version 2003/06/01 or later.
%%
%% This work has the LPPL maintenance status `maintained'.
%%
%% The Current Maintainer of this work is Boris Veytsman
%%
%    \begin{macrocode}
%<style>\NeedsTeXFormat{LaTeX2e}
%<*gobble>
\ProvidesFile{fitbox.dtx}
%</gobble>
%<style>\ProvidesPackage{fitbox}
%<*style>
[2015/02/02 v1.00 Fitting boxes on a page]
%    \end{macrocode}
%</style>
%<*gobble>
% \fi
% \CheckSum{0}
%
%
%% \CharacterTable
%%  {Upper-case    \A\B\C\D\E\F\G\H\I\J\K\L\M\N\O\P\Q\R\S\T\U\V\W\X\Y\Z
%%   Lower-case    \a\b\c\d\e\f\g\h\i\j\k\l\m\n\o\p\q\r\s\t\u\v\w\x\y\z
%%   Digits        \0\1\2\3\4\5\6\7\8\9
%%   Exclamation   \!     Double quote  \"     Hash (number) \#
%%   Dollar        \$     Percent       \%     Ampersand     \&
%%   Acute accent  \'     Left paren    \(     Right paren   \)
%%   Asterisk      \*     Plus          \+     Comma         \,
%%   Minus         \-     Point         \.     Solidus       \/
%%   Colon         \:     Semicolon     \;     Less than     \<
%%   Equals        \=     Greater than  \>     Question mark \?
%%   Commercial at \@     Left bracket  \[     Backslash     \\
%%   Right bracket \]     Circumflex    \^     Underscore    \_
%%   Grave accent  \`     Left brace    \{     Vertical bar  \|
%%   Right brace   \}     Tilde         \~} 
%
%\iffalse
%    \begin{macrocode}
\documentclass{ltxdoc}
\usepackage{hypdoc}
\PageIndex
\CodelineIndex
\RecordChanges
\EnableCrossrefs
\begin{document}
  \DocInput{fitbox.dtx}
\end{document}
%    \end{macrocode}
%</gobble> 
% \fi
% \MakeShortVerb{|}
% \GetFileInfo{fitbox.dtx}
% 
% 
% \title{Fit Box Package\thanks{\copyright 2015 Boris Veytsman}
% \thanks{This package was commissined by Neadwerx, \url{http://www.neadwerx.com/}}}
% \author{Boris Veytsman \thanks{borisv@lk.net}}
% \date{\filedate, \fileversion}
% \maketitle
%
% \begin{abstract}
%   The \textsl{fitbox} package allows a box (usually an
%   |\includegraphics| box to be fit on the page.  It scales the box
%   to the maximal allowed size within the user-set limits. If there
%   there is not enough space on the page, the box is moved to the
%   next one.
% \end{abstract}
%
% \tableofcontents
%
% \clearpage
%
%\section{Introduction}
%\label{sec:intro}
% 
% How often one puts a picture on a page only to see that \LaTeX\
% decides to move it to the next one because there is not enough
% space---while shaving a millimeter off the height would make the
% difference?  This package is intended to alleviate this difference.
% It uses several strategies to fit a picture on the page, and only if
% they fail, the picture is moved to the next one.
%
%\section{User Guide}
%\label{sec:ug}
%
%
%\subsection{Installation}
%\label{sec:ug_install}
%
% The installation of the class follows the usual
% practice~\cite{TeXFAQ} for \LaTeX{} packages:
% \begin{enumerate}
% \item Run \textsf{latex} on |fitbox.ins|.  This will produce the file
% |fitbox.sty|.
% \item Put the file |fitbox.sty| to
% the place where \LaTeX{} can find it (see
% \cite{TeXFAQ} or the documentation for your \TeX{}
% system).\label{item:install}
% \item Update the database of file names.  Again, see \cite{TeXFAQ}
% or the documentation for your \TeX{} system for the system-specific
% details.\label{item:update}
% \item The file |fitbox.pdf| provides the documentation for the
%   package
% \end{enumerate}
% As an alternative to items~\ref{item:install} and~\ref{item:update}
% you can just put the files in the working directory where your
% |.tex| file is.
% 
% 
%
%\subsection{Usage}
%\label{sec:usage}
%
% To use the package, add to the preamble of your document
% \begin{verbatim}
% \usepackage{fitbox}
% \end{verbatim}
% 
% \DescribeMacro{\fitbox}%
% The main command of the package is
% \cs{fitbox}\oarg{options}\marg{stuff}.  The \marg{stuff} will be
% typeset in a box according to the \oarg{options}.  In most cases
% \marg{stuff} is an \cs{includegraphics} command, but anything that
% fits into an LR-box can be typeset in this way.
%
% The \marg{stuff} is typeset in a box, and then the box is put on the
% page according to the following algorithm:
% \begin{enumerate}
% \item \TeX\ starts a new paragraph.
% \item The box is scaled up to the maximal dimensions specified by
% the user (while keeping the aspect ratio).
% \item If there is not enough space on the page to fit the box, the
%   latter is scaled down as neccessary, but no smaller than the
%   minimal dimensions specified by the user.
% \item If there is still not enough space, \TeX\ tries to enlarge the
% page up to the specified limit.
% \item If this also fails, \TeX\ starts a new page and fits the box
% there.  
% \end{enumerate}
% 
%
% \DescribeMacro{\fitboxset}%
% The options can be set individually for each \cs{fitbox} command, or
% globally using the command \cs{fitboxset}, for example,
% \begin{verbatim}
% \fitboxset{maxwidth=\textwidth, minwidth=\fitboxnatwidth}
% \end{verbatim}
% 
%
% \DescribeMacro{\fitboxnatwidth}%
% \DescribeMacro{\fitboxnatheight}%
% The options of the package use key-value interface.  Often
% the values are dimensions;  in these cases the special dimensions
% \cs{fitboxnatwidth} and \cs{fitboxnatheight} can be used;  they are
% equal to natural dimensions of the box.  Note that |height| and
% \cs{fitboxnatheight} are actually \emph{total heights}, including
% both height and depth of the corresponding boxes.  For example,
% \begin{verbatim}
% \fitboxeset{minheight=0.5\fitboxnatheight}
% \end{verbatim}
% means that the box cannot be scaled down more than 50\%.  
% 
% The following options are recognized:
% \begin{description}
% \item[maxheight:] The maximal total height of the box.  By default
% \cs{textheight}. 
% \item[maxwidth:] The maximal width of the box.  By default
% \cs{textwidth}. 
% \item[minheight:] The minimal height of the box.  By default
% \cs{fitboxnatheight}. 
% \item[minwidth:] The minimal width of the box.  By default
% \cs{fitboxnatwidth}. 
% \item[belowboxspace:] The height of the space that must be left
% below the box (e.g. for a caption).  By default zero.
% \item[maxenlargepage:] The maximal amount to add to the current
% page.  By default zero.
% \end{description}
% 
%
%\StopEventually{\clearpage
% \bibliography{fitbox}
% \bibliographystyle{unsrt}}
%
% \clearpage
% 
% \section{Implementation}
% \label{sec:implementation}
% 
%    \begin{macrocode}
%<*style>
%    \end{macrocode}
%
%
%\subsection{Setting up parameters}
%\label{sec:params}
%
% 
%   
% 
% \begin{macro}{\fitboxnatheight}
%   The total height of the box
%    \begin{macrocode}
\newdimen\fitboxnatheight
%    \end{macrocode}
%   
% \end{macro}
%
% \begin{macro}{\fitboxnatwidth}
%   The total width of the box
%    \begin{macrocode}
\newdimen\fitboxnatwidth
%    \end{macrocode}
%   
% \end{macro}
%
% 
%
% We use |xkeyval| interface:
%    \begin{macrocode}
\RequirePackage{xkeyval}
\define@cmdkeys{FTBX}{maxheight, minheight, maxwidth, minwidth,
  belowboxspace, maxenlargepage}
%    \end{macrocode}
%
% \begin{macro}{\fitboxset}
%   Setting everything
%    \begin{macrocode}
\def\fitboxset#1{\setkeys{FTBX}{#1}}
%    \end{macrocode}
%   
% \end{macro}
%
% The defaults
%    \begin{macrocode}
\fitboxset{maxheight=\textheight, minheight=\fitboxnatheight,
  maxwidth=\textwidth, minwidth=\fitboxnatwidth,
  belowboxspace=0pt, maxenlargepage=0pt}
%    \end{macrocode}
% 
%
%\subsection{Main command}
%\label{sec:main_command}
%
% \begin{macro}{\fitbox}
%   The main command
%    \begin{macrocode}
\newcommand\fitbox[2][]{%
  \fitboxset{#1}}
%    \end{macrocode}
%   
% \end{macro}
%
%    \begin{macrocode}
%</style>
%    \end{macrocode}
%\Finale
%\clearpage
%
%\PrintChanges
%\clearpage
%\PrintIndex
%
\endinput
