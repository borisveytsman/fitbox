% \iffalse
%
%% Copyright 2015-2019, Boris Veytsman <borisv@lk.net
%% This work may be distributed and/or modified under the
%% conditions of the LaTeX Project Public License, either
%% version 1.3 of this license or (at your option) any 
%% later version.
%% The latest version of the license is in
%%    http://www.latex-project.org/lppl.txt
%% and version 1.3 or later is part of all distributions of
%% LaTeX version 2003/06/01 or later.
%%
%% This work has the LPPL maintenance status `maintained'.
%%
%% The Current Maintainer of this work is Boris Veytsman
%%
%    \begin{macrocode}
%<style>\NeedsTeXFormat{LaTeX2e}
%<*gobble>
\ProvidesFile{fitbox.dtx}
%</gobble>
%<style>\ProvidesPackage{fitbox}
%<*style>
[2019/02/20 v1.02 Fitting boxes on a page]
%    \end{macrocode}
%</style>
%<*gobble>
% \fi
%
%
%% \CharacterTable
%%  {Upper-case    \A\B\C\D\E\F\G\H\I\J\K\L\M\N\O\P\Q\R\S\T\U\V\W\X\Y\Z
%%   Lower-case    \a\b\c\d\e\f\g\h\i\j\k\l\m\n\o\p\q\r\s\t\u\v\w\x\y\z
%%   Digits        \0\1\2\3\4\5\6\7\8\9
%%   Exclamation   \!     Double quote  \"     Hash (number) \#
%%   Dollar        \$     Percent       \%     Ampersand     \&
%%   Acute accent  \'     Left paren    \(     Right paren   \)
%%   Asterisk      \*     Plus          \+     Comma         \,
%%   Minus         \-     Point         \.     Solidus       \/
%%   Colon         \:     Semicolon     \;     Less than     \<
%%   Equals        \=     Greater than  \>     Question mark \?
%%   Commercial at \@     Left bracket  \[     Backslash     \\
%%   Right bracket \]     Circumflex    \^     Underscore    \_
%%   Grave accent  \`     Left brace    \{     Vertical bar  \|
%%   Right brace   \}     Tilde         \~} 
%
%\iffalse
%    \begin{macrocode}
\documentclass{ltxdoc}
\usepackage{hypdoc}
\PageIndex
\CodelineIndex
\RecordChanges
\EnableCrossrefs
\begin{document}
  \DocInput{fitbox.dtx}
\end{document}
%    \end{macrocode}
%</gobble> 
% \fi
% \MakeShortVerb{|}
% \GetFileInfo{fitbox.dtx}
% 
% 
% \title{Fit graphics on a page\thanks{\copyright 2015--2019 Boris Veytsman}
% \thanks{This package was commissined by Neadwerx, \url{http://www.neadwerx.com/}}}
% \author{Boris Veytsman \thanks{borisv@lk.net}}
% \date{\filedate, \fileversion}
% \maketitle
%
% \begin{abstract}
%   The \textsl{fitbox} package allows a box (usually an
%   |\includegraphics| box) to fit on the page.  It scales the box
%   to the maximal allowed size within the user-set limits. If there
%   there is not enough space on the page, the box is moved to the
%   next one.
% \end{abstract}
%
% \tableofcontents
%
% \clearpage
%
%\section{Introduction}
%\label{sec:intro}
% 
% How often one puts a picture on a page only to see that \LaTeX\
% decides to move it to the next one because there is not enough
% space---while shaving a millimeter off the height would make the
% difference?  This package is intended to alleviate this difference.
% It uses several strategies to fit a picture on the page, and only if
% they fail, the picture is moved to the next one.
%
%\section{User Guide}
%\label{sec:ug}
%
%
%\subsection{Installation}
%\label{sec:ug_install}
%
% The installation of the class follows the usual
% practice~\cite{TeXFAQ} for \LaTeX{} packages:
% \begin{enumerate}
% \item Run \textsf{latex} on |fitbox.ins|.  This will produce the file
% |fitbox.sty|.
% \item Put the file |fitbox.sty| to
% the place where \LaTeX{} can find it (see
% \cite{TeXFAQ} or the documentation for your \TeX{}
% system).\label{item:install}
% \item Update the database of file names.  Again, see \cite{TeXFAQ}
% or the documentation for your \TeX{} system for the system-specific
% details.\label{item:update}
% \item The file |fitbox.pdf| provides the documentation for the
%   package
% \end{enumerate}
% As an alternative to items~\ref{item:install} and~\ref{item:update}
% you can just put the files in the working directory where your
% |.tex| file is.
% 
% 
%
%\subsection{Usage}
%\label{sec:usage}
%
% To use the package, add to the preamble of your document
% \begin{verbatim}
% \usepackage{fitbox}
% \end{verbatim}
% 
% \DescribeMacro{\fitbox}%
% The main command of the package is
% \cs{fitbox}\oarg{options}\marg{stuff}.  The \marg{stuff} will be
% typeset in a box according to the \oarg{options}.  In most cases
% \marg{stuff} is an \cs{includegraphics} command, but anything that
% fits into an LR-box can be typeset in this way.
%
% The \marg{stuff} is typeset in a box, and then the box is put on the
% page according to the following algorithm:
% \begin{enumerate}
% \item \TeX\ starts a new paragraph.
% \item The box is scaled up to the maximal dimensions specified by
% the user (while keeping the aspect ratio).
% \item If there is not enough space on the page to fit the box, the
%   latter is scaled down as neccessary, but no smaller than the
%   minimal dimensions specified by the user.
% \item If there is still not enough space, \TeX\ tries to enlarge the
% page up to the specified limit.
% \item If this also fails, \TeX\ starts a new page and fits the box
% there.  
% \end{enumerate}
%
% \DescribeMacro{\fitbox*}%
% The starred version \cs{fitbox*} is intended to be used inside
% floats, where page length is not well defined.  This command uses
% only the first two steps of the alogithm above, scaling the box up
% to the maximal dimensions provided by the user.
%
% \DescribeMacro{\fitboxset}%
% The options can be set individually for each \cs{fitbox} command, or
% globally using the command \cs{fitboxset}, for example,
% \begin{verbatim}
% \fitboxset{maxwidth=\textwidth, minwidth=\fitboxnatwidth}
% \end{verbatim}
% 
%
% \DescribeMacro{\fitboxnatwidth}%
% \DescribeMacro{\fitboxnatheight}%
% The options of the package use key-value interface.  Often
% the values are dimensions;  in these cases the special dimensions
% \cs{fitboxnatwidth} and \cs{fitboxnatheight} can be used;  they are
% equal to natural dimensions of the box.  Note that |height| and
% \cs{fitboxnatheight} are actually \emph{total heights}, including
% both height and depth of the corresponding boxes.  For example,
% \begin{verbatim}
% \fitboxset{minheight=0.5\fitboxnatheight}
% \end{verbatim}
% means that the box cannot be scaled down more than 50\%.  
% 
% The following options are recognized by the \cs{fitbox} command:
% \begin{description}
% \item[maxheight:] The maximal total height of the box.  By default
% \cs{textheight}. 
% \item[maxwidth:] The maximal width of the box.  By default
% the size of the current box to be constructed. 
% \item[minheight:] The minimal height of the box.  By default
% \cs{fitboxnatheight}. 
% \item[minwidth:] The minimal width of the box.  By default
% \cs{fitboxnatwidth}. 
% \item[belowboxspace:] The height of the space that must be left
% below the box (e.g. for a caption).  By default zero.
% \item[maxenlargepage:] The maximal amount to add to the current
% page.  By default zero.
% \end{description}
%
% \DescribeMacro{\SetFitboxLayout}%
% Since version 1.02 the package added the functionality of
% \textsl{FigSize} package~\cite{FigSize}.  The latter has a handy
% command \cs{SetFigLayout}, which can be used to scale all figures on
% a page.  This package uses the command \cs{SetFitboxLayout} that
% somewhat mimics \cs{SetFigLayout}.  The macro has the arguments
% \cs{SetFitboxLayout}\oarg{keys}\marg{rows}\marg{columns}, and helps
% to scale the boxes into a grid.  However, unlike \textsl{FigSize}
% package, we do not redefine \cs{includegraphics} command: you should
% use \cs{fitbox} for your graphics.  The command \cs{SetFitboxLayout}
% is just \cs{fitboxset} with special values of the parameters.  These
% commands can be mixed, with the latest command overriding the
% previous ones.
%
% The following keys are recognized by the \cs{SetFitboxLayout} command:
% \begin{description}
% \item[colsep:] the distance between the columns; by default the size
% of \cs{quad}.
% \item[colsepexpr:] if set, the width of the expression becomes
% \texttt{colsep}.  For example, \path{colsepexpr=\quad} makes
% separation between the columns the size of \cs{quad}.
% \item[maincapheight:] The height of the main caption.  By default,
% $\cs{abovecaptionskip}+\cs{belowcaptionskip}+
% n\cs{baselineskip})$ 
% where $n$ is the expected number of lines in the main caption (by
% default 1).
% \item[maincaplines:] The number of lines in the main caption.  By
% default 1.
% \item[subcapheight:] The height of the subcaption.  By default,
%   $\cs{abovecaptionskip}+\cs{belowcaptionskip}+m\cs{baselineskip}$
%   where $m$ is the expected number of lines in the subcaptions (by
%   default 0).
% \item[subcaplines:] The number of lines in the subcaptions.  By
% default 0.
% \end{description}
% 
%
%\StopEventually{\clearpage
% \bibliography{fitbox}
% \bibliographystyle{unsrt}}
%
% \clearpage
% 
% \section{Implementation}
% \label{sec:implementation}
% 
%    \begin{macrocode}
%<*style>
%    \end{macrocode}
%
%
%\subsection{Setting up parameters}
%\label{sec:params}
%
% 
%   
% 
% \begin{macro}{\fitboxnatheight}
%   The total height of the box
%    \begin{macrocode}
\newdimen\fitboxnatheight
%    \end{macrocode}
%   
% \end{macro}
%
% \begin{macro}{\fitboxnatwidth}
%   The total width of the box
%    \begin{macrocode}
\newdimen\fitboxnatwidth
%    \end{macrocode}
%   
% \end{macro}
%
% 
% \changes{v1.01}{2019/02/20}{New keys for \cs{SetFitboxLayout}}
% We use |xkeyval| interface:
%    \begin{macrocode}
\RequirePackage{xkeyval}
\define@cmdkeys{FTBX}{maxheight, minheight, maxwidth, minwidth,
  belowboxspace, maxenlargepage, colsep, maincapheight,
  subcapheight}
%    \end{macrocode}
%
% Setting some keys changes other keys
%    \begin{macrocode}
\define@key{FTBX}{colsepexpr}{%
  \setbox\@tempboxa=\hbox{#1}%
  \edef\cmdKV@FTBX@colsep{\wd\@tempboxa}}
\define@key{FTBX}{maincaplines}{%
  \@tempdima=\z@
  \advance\@tempdima by #1\baselineskip\relax
  \ifnum#1>0\relax
     \advance\@tempdima by \parskip
  \fi 
  \advance\@tempdima by \abovecaptionskip
  \advance\@tempdima by \belowcaptionskip
  \edef\cmdKV@FTBX@maincapheight{\the\@tempdima}}
\define@key{FTBX}{subcaplines}{%
  \@tempdima=\z@
  \advance\@tempdima by #1\baselineskip\relax
  \ifnum#1>0\relax
     \advance\@tempdima by \parskip
  \fi 
  \advance\@tempdima by \abovecaptionskip
  \advance\@tempdima by \belowcaptionskip
  \edef\cmdKV@FTBX@subcapheight{\the\@tempdima}}  
%    \end{macrocode}
% 
%
% \begin{macro}{\fitboxset}
%   Setting everything
%    \begin{macrocode}
\def\fitboxset#1{\setkeys{FTBX}{#1}}
%    \end{macrocode}
%   
% \end{macro}
%
%\changes{v1.01}{2019/02/10}{Changed maximal width}
% The defaults
%    \begin{macrocode}
\fitboxset{maxheight=\textheight, minheight=\fitboxnatheight,
  maxwidth=\hsize, minwidth=\fitboxnatwidth,
  belowboxspace=0pt, maxenlargepage=0pt, colsepexpr={aaa},
  maincaplines=1, subcaplines=0}
%    \end{macrocode}
% 
%
%\subsection{Main command}
%\label{sec:main_command}
%
% \begin{macro}{\FTBX@box}
%   The box which will held the stuff to be typeset
%    \begin{macrocode}
\newbox\FTBX@box
%    \end{macrocode}
%   
% \end{macro}
%
% \begin{macro}{\FTBX@desired@maxheight}
%   The desired maximal height
%    \begin{macrocode}
\newdimen\FTBX@desired@maxheight
%    \end{macrocode}
% \end{macro}
%
%
% \begin{macro}{\FTBX@desired@minheight}
% The desired minimal height
%    \begin{macrocode}
\newdimen\FTBX@desired@minheight
%    \end{macrocode}
% \end{macro}
%
% \begin{macro}{\FTBX@available@height}
% The desired available height
%    \begin{macrocode}
\newdimen\FTBX@available@height
%    \end{macrocode}
% \end{macro}
%
% \begin{macro}{\fitbox}
% \changes{v1.02}{2019/02/18}{Split into normal and starred version}
% The main command
%    \begin{macrocode}
\def\fitbox{\@ifstar\@@fitbox\@fitbox}
%    \end{macrocode}
% 
% \end{macro}
%
% \begin{macro}{\@fitbox}
% \changes{v1.01}{2019/02/10}{Do not force width adjustment}
% \changes{v1.02}{2019/02/17}{\cs{noident} instead of \cs{leavevmode}
% (suggested by Frank Mittelbach)}
% \changes{v1.02}{2019/02/18}{Split into normal and starred version}
%   The main command---``normal'' version
%    \begin{macrocode}
\newcommand\@fitbox[2][]{\noindent
  \fitboxset{#1}%
  \setbox\FTBX@box=\hbox{#2}%
  \fitboxnatwidth=\wd\FTBX@box\relax
  \fitboxnatheight=\ht\FTBX@box\relax
  \advance\fitboxnatheight by \dp\FTBX@box\relax
  % Checking the sizes
  \expandafter\ifdim\cmdKV@FTBX@minwidth>\columnwidth\relax
      \PackageWarning{fitbox}{Minimal width is too large.  Consider
        changing it to \the\hsize}%
  \fi
  \expandafter\ifdim\cmdKV@FTBX@maxwidth>\hsize\relax
      \PackageWarning{fitbox}{Desired width is too large.  Consider
        changing it to \the\hsize}%
  \fi
  \expandafter\ifdim\cmdKV@FTBX@minheight>\textheight\relax
      \PackageWarning{fitbox}{Minimal height is too large.  
        Consider changing it to \the\textheight}%
  \fi
  \expandafter\ifdim\cmdKV@FTBX@maxheight>\textheight\relax
      \PackageWarning{fitbox}{Desired height is too large.
        Consider changing it to \the\textheight}%
  \fi
  % Calculating the minimal and maximal height
  \Gscale@div{\@FTBX@tempa}{\cmdKV@FTBX@maxwidth}{\fitboxnatwidth}%
  \FTBX@desired@maxheight=\@FTBX@tempa\fitboxnatheight\relax
  \expandafter\ifdim\cmdKV@FTBX@maxheight<\FTBX@desired@maxheight\relax
     \expandafter\FTBX@desired@maxheight=\cmdKV@FTBX@maxheight\relax
  \fi
  \Gscale@div{\@FTBX@tempa}{\cmdKV@FTBX@minwidth}{\fitboxnatwidth}%
  \FTBX@desired@minheight=\@FTBX@tempa\fitboxnatheight\relax
  \expandafter\ifdim\cmdKV@FTBX@minheight>\FTBX@desired@minheight\relax
     \expandafter\FTBX@desired@minheight=\cmdKV@FTBX@minheight\relax
  \fi
  \ifdim\FTBX@desired@minheight>\FTBX@desired@maxheight\relax
      \PackageWarning{fitbox}{Desired min scale exceeds desired max
        scale.}% 
  \fi  
  \FTBX@available@height=\pagegoal\relax
  \ifdim\FTBX@available@height>\vsize\relax
    \FTBX@available@height=\vsize
  \fi
  \advance\FTBX@available@height by -\pagetotal\relax
  \advance\FTBX@available@height by -\cmdKV@FTBX@belowboxspace\relax
  \advance\FTBX@available@height by -\baselineskip\relax
  \ifdim\FTBX@desired@maxheight>\FTBX@available@height\relax
     \ifdim\FTBX@available@height<\FTBX@desired@minheight\relax 
        \@tempdima=\FTBX@desired@minheight\relax
        \advance\@tempdima by
        -\FTBX@available@height\relax
        \expandafter\ifdim\cmdKV@FTBX@maxenlargepage<\@tempdima\relax
          \newpage
          \resizebox*{!}{\FTBX@desired@maxheight}{\box\FTBX@box}%
        \else
          \enlargethispage{\@tempdima}%
          \resizebox*{!}{\FTBX@desired@minheight}{\box\FTBX@box}%
        \fi
      \else 
       \resizebox*{!}{\FTBX@available@height}{\box\FTBX@box}%
      \fi
  \else
     \resizebox*{!}{\FTBX@desired@maxheight}{\box\FTBX@box}%
  \fi
}
%    \end{macrocode}
%   
% \end{macro}
%
% \begin{macro}{\@@fitbox}
% \changes{v1.01}{2019/02/10}{Do not force width adjustment}
% \changes{v1.02}{2019/02/17}{\cs{noident} instead of \cs{leavevmode}
% (suggested by Frank Mittelbach)}
% \changes{v1.02}{2019/02/18}{Split into normal and starred version}
%   The main command---``starred'' version.  Simplified computations.
%    \begin{macrocode}
\newcommand\@@fitbox[2][]{\noindent
  \fitboxset{#1}%
  \setbox\FTBX@box=\hbox{#2}%
  \fitboxnatwidth=\wd\FTBX@box\relax
  \fitboxnatheight=\ht\FTBX@box\relax
  \advance\fitboxnatheight by \dp\FTBX@box\relax
  % Checking the sizes
  \expandafter\ifdim\cmdKV@FTBX@minwidth>\columnwidth\relax
      \PackageWarning{fitbox}{Minimal width is too large.  Consider
        changing it to \the\hsize}%
  \fi
  \expandafter\ifdim\cmdKV@FTBX@maxwidth>\hsize\relax
      \PackageWarning{fitbox}{Desired width is too large.  Consider
        changing it to \the\hsize}%
  \fi
  \expandafter\ifdim\cmdKV@FTBX@minheight>\textheight\relax
      \PackageWarning{fitbox}{Minimal height is too large.  
        Consider changing it to \the\textheight}%
  \fi
  \expandafter\ifdim\cmdKV@FTBX@maxheight>\textheight\relax
      \PackageWarning{fitbox}{Desired height is too large.
        Consider changing it to \the\textheight}%
  \fi
  % Calculating the minimal and maximal height
  \Gscale@div{\@FTBX@tempa}{\cmdKV@FTBX@maxwidth}{\fitboxnatwidth}%
  \FTBX@desired@maxheight=\@FTBX@tempa\fitboxnatheight\relax
  \expandafter\ifdim\cmdKV@FTBX@maxheight<\FTBX@desired@maxheight\relax
     \expandafter\FTBX@desired@maxheight=\cmdKV@FTBX@maxheight\relax
  \fi
  \Gscale@div{\@FTBX@tempa}{\cmdKV@FTBX@minwidth}{\fitboxnatwidth}%
  \FTBX@desired@minheight=\@FTBX@tempa\fitboxnatheight\relax
  \expandafter\ifdim\cmdKV@FTBX@minheight>\FTBX@desired@minheight\relax
     \expandafter\FTBX@desired@minheight=\cmdKV@FTBX@minheight\relax
  \fi
  \ifdim\FTBX@desired@minheight>\FTBX@desired@maxheight\relax
      \PackageWarning{fitbox}{Desired min scale exceeds desired max
        scale.}% 
  \fi  
  \resizebox*{!}{\FTBX@desired@maxheight}{\box\FTBX@box}%
}
%    \end{macrocode}
%   
% \end{macro}
%
%
%\subsection{Multi-figure layout}
%\label{sec:layout}
%
%
%
% \begin{macro}{\SetFitboxLayout}
% \changes{v1.02}{2019/02/17}{New macro}
% \changes{v1.02}{2019/02/20}{Key-value interface}
% A macro styles after \cs{SetFigsizeLayout} of \textsl{FigSize}
% package.
%    \begin{macrocode}
\newcommand\SetFitboxLayout[3][]{%
  \fitboxset{#1}%
  \@tempdima=\textheight
  \advance\@tempdima by -\cmdKV@FTBX@maincapheight\relax
  \divide\@tempdima by #2\relax
  \advance\@tempdima by -\cmdKV@FTBX@subcapheight\relax
  \edef\cmdKV@FTBX@maxheight{\the\@tempdima}%
  \@tempdima=\textwidth
  \@tempdimb=\cmdKV@FTBX@colsep\relax
  \advance\@tempdima by -#3\@tempdimb\relax
  \advance\@tempdima by \@tempdimb\relax
  \divide\@tempdima by #3\relax
  \edef\cmdKV@FTBX@maxwidth{\the\@tempdima}%
}
%    \end{macrocode}
% 
% \end{macro}
%
%    \begin{macrocode}
%</style>
%    \end{macrocode}
%\Finale
%\clearpage
%
%\PrintChanges
%\clearpage
%\PrintIndex
%
\endinput
